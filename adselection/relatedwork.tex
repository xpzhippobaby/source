\section{Related Work}
Techniques proposed in this paper are related to keyword suggestion
and query expansion.



Because the average length of queries are around
2~\cite{baeza:searchengines}, query expansion has been thought as an
effective way to resolve word mismatching
problems~\cite{cui:querylogs}.
Existing approaches mainly focus on expanding queries with various
external resources including organic search
results~\cite{broder:relevancefeedback, joshi:engineadvertising}, user
behavior
data~\cite{cui:querylogs,broder:webknowledge,fuxman:keywordgeneration}
and bidding relationships between phrases and
ads~\cite{wang:advertisementsearch}.
%Modern search engines can return decent organic search
%results even the search query is very short.
%Fertilizing queries with algorithmic search results would bring in
%unacceptable latency for sponsored advertising and can't be applied
%real-time.
%User behavior data including click-through data, query session data
%may capture the semantic relevance between search phrases.
Ricardo et al~\cite{baeza:searchengines} propose to represent queries
by aggregation of terms-weight vectors of their clicked URLs, which
leverages both organic search results and user behavior data.
However, in most cases, there is no adequate user behavior data for
tail queries that are individually rare but make up a significant
portion of the query volume~\cite{broder:sponsoredsearch}.



%Text snippets are firstly mapped into their concept hierarchy.
%Then the similarity between original text snippets are measured by
%similarity between their corresponding set of concepts using
%conventional metrics.
%To fill the gap between the keywords chosen by advertisers and their
%potential customers, keyword suggestion technology is employed.
%Chen\cite{chen:concepthierarchy} exploits the semantic knowledge
%derived from open directory project(ODP)\cite{odp:tool}.
%Query is firstly mapped into their concept hierarchy.
%Then the new keywords are suggested according to the concept
%information of original keyword.
%As they pointed, their approach is directly affected by the quality
%and the coverage of the concept hierarchy.
%However, they consider categories in ODP as the concepts in their
%hierarchy and unfortunately there are only about 150,446 ODP
%categories.
%Cao\cite{cao:sessiondata} propose clustering queries based on
%click-through data and making context-aware suggestion by mining
%session data.\\
%For search engines, the most important task for them is to match
%users' submitted queries to relevant ads.
%Traditionally, sponsored search employed standard information
%retrieval techniques using the bag of words(BoW) approach.
%Since BoW model ignores relations between words and can not access the
%semantics behind text.
%It is challenging to identify relevant ads for user's query when the
%query is very short.
%In fact, web users usually issue very short queries to search engines
%whose average length is less than two words\cite{wen:searchengine}.
%Besides, users typically choose terms intended to achieve optimal web
%search results rather than optimal ads and ads themselves are also
%very short in most cases\cite{broder:relevancefeedback}.
%Classical works related to such task employed various resources as
%additional knowledge to enrich the features of short queries.
%Using organic search results to provide greater context for the short
%texts are widely accepted\cite{broder:relevancefeedback}.
%User behavior data including click-through data, user query session
%and bidding information are used to derive information about
%similarity between terms or short texts.
%Cui\cite{cui:querylogs} exploits correlations among terms in documents
%and user queries mined from user logs.
%Wang\cite{wang:advertisementsearch} propose clustering queries based
%on the bipartite graph between queries and ads.
%Approaches making full use of taxonomy work in explicit semantic
%analysis(ESA)\cite{gabrilovich:semanticanalysis} manner.
%There are various taxonomies are used for query expansion including
%Wikipedia\cite{hu:withwikipedia}, open directory
%project(ODP)\cite{chen:concepthierarchy} and
%WordNet\cite{voorhees:semanticrelations}.
%Some techniques combining various additional resources to be used are
%proposed.
%Broder\cite{broder:sponsoredsearch} enrichs the features of head
%queries with organic search results offline and online expands rare
%queries by retrieve related head queries from a vector space model
%using textual features as well as semantic-class features.
%Query expansion research has more than three decades history and we
%can't introduce all the classical works here.
%\cite{carpineto:informationretrieval} is a detailed and genealogical
%survey on query expansion.\\
More fundamentally, the key problem of many queries processing tasks
and keyword research is measuring similarity between short texts.
Previous work on measuring text semantic similarity has focused mainly on either large
documents or individual words~\cite{mihalcea:semanticsimilarity}.
Since both queries and bid phrases are too short to draw reliable
statistical conclusion, LSA approaches~\cite{blei:dirichletallocation,
    deerwester:semanticanalysis, hofmann:semanticindexing} are not an
    effective way to index short text snippets.
Besides corpus-based approaches, existing approaches for measuring
short text semantic similarity are largely
knowledge-based~\cite{gabrilovich:semanticanalysis,hu:withwikipedia,chen:concepthierarchy,song:probabilisticknowledgebase}.
Knowledgebase including WordNet~\cite{wordnet:tool}, open directory
project(ODP)~\cite{odp:tool} and wikipedia\cite{wiki:tool} are
exploited to enrich short text snippets at semantic level.
As Chen et al~\cite{chen:concepthierarchy} pointed, the performance of
knowledge-based approaches are strikingly affected by the accuracy and the coverage
of used concept hierarchy.
%and unfortunately there are only about
%150446 ODP categories which are considered as concepts in their
%approach.
Our approach model and rank short text with the help of a probabilistic taxonomy
named Probase~\cite{wu:manyconcepts} which, to the best of our
knowledge, contains the largest concept space compared with those
knowledgebases adopted by existing approaches.
Besides, most of these knowledge-based approaches map short text
snippets into their concept space based on co-occurrence of terms
within the content of original text and terms appearing in
corresponding articles or web pages of concepts.
Considering that queries and bid phrases are very short, such strategy
is not reasonable.
