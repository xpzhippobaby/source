\section{Flight Analysis}
In reality, search engine preprocesses a list of queries offline and
builds inverted index for the resulting expansions(query as term and
    its expansion as corresponding posting list).
At runtime, search engine uses the resulting expansions to find
relevant bid keywords for submitted query.
When there is enough traffic volumne to ensure computed metrics(e.g.,
        click through rate(CTR), cost per click(CPC), etc.) reliable,
     search engine will analysis and prune those bad expansion.
For example, a certain bid keyword is high-ranking in the expansion
for a certain query.
However, the CTR for ads retrieved through this bid keyword is so low.
This bid keyword may be not so relevant to that query and search
engine will eliminate it from that query's expansion in the next
version.
Although pruning is not an elegant approach to improve effectiveness
of sponsored search, it is an important and indespensable step of
sponsored search adopted by many mainstream search engines.\\
Different from latent semantic analysis approaches which represent
documents as distribution over latent topics and topics as
distributions over vocabulary of indexed corpus, our approach
represents documents as explicit topics where topics are concepts used
in our daily life.
Such a representation representation of the meaning behind any text is
easy to explain to human users\cite{gabrilovich:semanticanalysis}.
By our explicit representation of semantic meaning, we are able to not
only prune resulting expansion based on conventioinal metrics but also
infer and analysis more detailed problem behind the sympton such as
which topics(concepts) contain commercial intent, which domains our
approach can't not generate useful expansions.
For example, when we find that CTRs of expansions for queries related
to concept ``hotel'' are lower than other topics(concepts).
We can collect these queries and design vertical search engine for
this specific domain.

