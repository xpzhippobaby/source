\subsection{Information Theory}
When we observe a specific value of a random variable $X=x$, the
information gain is:
\begin{definition}
\begin{equation}
h(x)=-\log_{2}\Pr(x)
\label{eqn:ig}
\end{equation}
\end{definition}
which is a monotonic function of $\Pr(x)$ and reflects the ``degree of
surprise''. For independent random variables $X,Y$,
    $\Pr(x,y)=\Pr(x)\Pr(y)$, thus the information gain
    $h(x,y)=\log_{2}\Pr(x,y)=\log_{2}\Pr(x)+\log_{2}\Pr(y)=h(x)+h(y)$
    which satisfies our intuition.


\emph{Entropy} can be viewed as the average amount of information to transmit
a random variable. Thus is defined as the expectation of
\eqref{eqn:ig} with respect to the distribution $\Pr(X)$:
\begin{definition}
\begin{equation}
H[X]=-\sum_{x}\Pr(x)\log_{2}\Pr(x)
\label{eqn:entropy}
\end{equation}
\end{definition}
A deeper interpretation as a measure of disorder:
\putansline{6}{1}


The more evenly a distribution spreads, the higher its entropy is.
Besides, the \emph{noiseless coding theorem} states that:
\begin{conclusion}
The entropy is a lower bound on the average number of bits needed to transmit
the state of a random variable. We can arrive the lower bound by
choosing an efficient coding scheme.
\end{conclusion}
The maximum entropy configuration can be found by maximizing the following
equation which includes the normalization constraint on the
probability distribution:
\begin{equation}
\tilde{H}=-\sum_{i=1}^{M}p_{i}\ln(p_{i})+\lambda(\sum_{i=1}^{M}p_{i}-1)
\label{eqn:tildeh}
\end{equation}
where the $\lambda$ is a Lagrangian multiplier and doesn't need to be 
determined. To find stationary points, we check its derivatives:
\begin{gather}
\frac{\partial{}\tilde{H}}{\partial{}p_{i}}=-(\ln(p_{i})+1)+\lambda=0\text{~for~}i=1,\ldots,M\notag\\
\sum_{i=1}^{M}p_{i}-1=0\notag
\end{gather}
We have stationary $p_{i}=\frac{1}{M},i=1,\ldots,M$ and its second
derivative is:
\begin{equation}
\frac{\partial{}\tilde{H}}{\partial{}p_{i}\partial{}p_{j}}=-I_{ij}\frac{1}{p_i}=-I_{ij}M
\label{eqn:sd}
\end{equation}
Since $M>0$, the second derivative is negative definite (diagonal
        matrix's eigenvalues are elements on its diagonal), thus the
stationary point is indeed a maximum.


The entropy for continuous variable can't be elegantly defined and we
use the concept---differential entropy:
\begin{definition}
\begin{equation}
H[X]=-\int{}\Pr(x)\ln{}\Pr(x)\text{d}x
\label{eqn:diffentropy}
\end{equation}
\end{definition}

We find the maximum of $H[x]$ using Lagrange multipliers with 
normalization constraint of probability distribution and contraints over the first and
second moments of $\Pr(X)$:
\begin{gather}
\lambda_{1}(\int_{-\infty}^{\infty}\Pr(x)\text{d}x-1)\notag\\
\lambda_{2}(\int_{-\infty}^{\infty}x\Pr(x)\text{d}x-\mu)\notag\\
\lambda_{3}(\int_{-\infty}^{\infty}(x-\mu)^{2}\Pr(x)\text{d}x-\sigma^{2})\notag
\end{gather}
Using the calculus of variations and set the derivative of the
functional to zero giving us:
\putansline{6}{2}
\begin{equation}
\Pr(X)=\exp{}\{-1+\lambda_{1}+\lambda_{2}X+\lambda_{3}(X-\mu)^2\}
\label{eqn:cvresult}
\end{equation}
back substitution of \eqref{eqn:cvresult} into the three constraint
equations gives us:
\putansline{6}{3}
\begin{equation}
\Pr(X)=\frac{1}{(2\pi{}\sigma^{2})^{\frac{1}{2}}}\exp{}\{-\frac{(X-\mu)^{2}}{2\sigma^{2}}\}
\label{eqn:uniguassian}
\end{equation}
so we have:
\begin{conclusion}
The maximum entropy configuration for continuous random variable is the Guassian.
\end{conclusion}
Then we evaluate its differential entropy. $\ln{}\Pr(X)=\frac{-1}{2}\ln{}(2\pi{}\sigma^2)-\frac{(X-\mu)^2}{2\sigma^2}$
    and
    $-\int_{-\infty}^{\infty}\Pr(X)\frac{-1}{2}\ln(2\pi{}\sigma^2)\text{d}X=\frac{1}{2}\ln(2\pi\sigma^2)$.
    So the difficulty is to calculate
\begin{equation}
\begin{split}
&\frac{1}{(2\pi\sigma^2)^{\frac{1}{2}}}\int_{-\infty}^{\infty}\exp{}\{\frac{-(X-\mu)^2}{2\sigma^2}\}\frac{(X-\mu)^2}{2\sigma^2}\text{d}X\\
=&\frac{(2\sigma^2)^{\frac{1}{2}}}{(2\pi\sigma^2)^{\frac{1}{2}}}\int_{-\infty}^{\infty}\exp{}\{\frac{-(X-\mu)^2}{2\sigma^2}\}\frac{(X-\mu)^2}{2\sigma^2}\text{d}\frac{(X-\mu)}{(2\sigma^2)^{\frac{1}{2}}}
\end{split}
\label{eqn:secondterm}
\end{equation}
By subsection integral method, we know that:
\begin{equation}
\begin{split}
\int_{-\infty}^{\infty}x^{2}\exp{}(-x^2)\text{d}x=&\int_{-\infty}^{\infty}\frac{\text{d}\exp{}(-x^2)}{\text{d}x}\frac{-x}{2}\text{d}x\\
=&\exp{}(-x^2)\frac{-x}{2}\vert_{-\infty}^{\infty}-\int_{-\infty}^{\infty}\exp{}(-x^2)\frac{\text{d}\frac{-x}{2}}{\text{d}x}\text{d}x\\
=&0+\frac{1}{2}\int_{-\infty}^{\infty}\exp{}(-x^2)\text{d}x\\
=&\frac{\pi^{\frac{1}{2}}}{2}
\end{split}
\label{eqn:ssint}
\end{equation}
Thus, based on \eqref{eqn:ssint} and \eqref{eqn:secondterm}, we have
$H[X]=\frac{1}{2}\{1+\ln{}(2\pi\sigma^2)\}$.


Suppose we have a joint distribution $\Pr(X,Y)$. If we observe a
specific value $X=x$, then the additional information needed to
specify the corresponding value of $Y=y$ is given by
$-\ln{}\Pr(y\vert{}x)$. Intuitively, $h(x,y)=h(x)+h(y\vert{}x)$, by
the production rule of probability,
    $-\ln{}\Pr(x,y)=-\ln{}\{\Pr(x)\Pr(y\vert{}x)\}=-\ln{}\Pr(x)-\ln{}\Pr(y\vert{}x)$,
    which comfirms that our definition makes sense.
    
    
Then we define the \emph{conditional entropy} of $Y$ given $X$ as the
average additional information needed to specify $Y$:
\begin{definition}
\begin{equation}
H[Y\vert{}X]=-\int{}\int{}\Pr(X,Y)\ln{}\Pr(Y\vert{}X)\text{d}Y\text{d}X
\label{eqn:conditionalentropy}
\end{equation}
\end{definition}
Then we confirm that $H[X,Y]=H[Y\vert{}X]+H[X]$:
\begin{equation}
\begin{split}
H[X,Y]=&-\int{}\int{}\Pr(X,Y)\ln{}\Pr(X,Y)\text{d}X\text{d}Y\\
=&-\int{}\int{}\Pr(X,Y)\ln{}\Pr(Y\vert{}X)+\Pr(X,Y)\ln{}\Pr(X)\text{d}X\text{d}Y\\
=&H[Y\vert{}X]-\int{}\ln{}\Pr(X)\text{d}X\int{}\Pr(X,Y)\text{d}Y\\
=&H[Y\vert{}X]-\int{}\Pr(X)\ln{}\Pr(X)\text{d}X\\
=&H[Y\vert{}X]+H[X]
\end{split}
\label{eqn:condhhh}
\end{equation}
\subsubsection{Relative entropy and mutual information}
\begin{definition}
\emph{Convex} functions are functions that satisfy the following
inequality:
\begin{equation}
f(\lambda{}a+(1-\lambda)b)\leq{}\lambda{}f(a)+(1-\lambda)f(b)
\label{eqn:convexity}
\end{equation}
where $0\leq{}\lambda{}\leq{}1$.
\end{definition}


A convex function satisfies the \emph{Jensen's inequality}:
\begin{equation}
f(\sum_{i=1}^{M}\lambda_{i}x_{i})\leq{}\sum_{i=1}^{M}\lambda_{i}f(x_i)
\label{eqn:jensen}
\end{equation}
where $\lambda_{i}\geq{}0$ and $\sum_{i=1}^{M}\lambda_{i}=1$. We can
prove the above inequality by induction:
\begin{proof}
\begin{equation}
\sum_{i=1}^{N+1}\lambda_{i}f(x_i)=(1-\lambda_{N+1})\sum_{i=1}^{N}\frac{\lambda_{i}}{1-\lambda_{N+1}}f(x_i)+\lambda_{N+1}f(x_{N+1})
\label{eqn:tjensen}
\end{equation}
$\because$ \eqref{eqn:jensen} is valid for previous $N$ cases
$\therefore{}$ \eqref{eqn:tjensen}
$\geq{}(1-\lambda_{N+1})f(\sum_{i=1}^{N}\frac{\lambda_{i}}{1-\lambda_{N+1}}x_{i})+\lambda_{N+1}f(x_{N+1})$\\
$\because{}\quad{}f(\cdot)$ is a convex function $\therefore{}$ we
have:
\begin{equation}
(1-\lambda_{N+1})f(\sum_{i=1}^{N}\frac{\lambda_{i}}{1-\lambda_{N+1}}x_{i})+\lambda_{N+1}f(x_{N+1})\geq{}f(\sum_{i=1}^{N+1}\lambda_{i}x_{i})
\label{eqn:t2jensen}
\end{equation}
Thus, we derive the $N+1$ case from previous case.
\end{proof}
We can interpret $\lambda_{i}$ as the probability over $X=x_{i}$, then
\eqref{eqn:jensen} can be written as:
\begin{equation}
f(E[X])\leq{}E[f(X)]
\label{eqn:expectationjensen}
\end{equation}
which is also valid for continuous case.


Suppose a random variable $X$ is transmitted and its distribution
$p(X)$ is unknown. We use distribution $q(X)$ to approximate it. Thus
the average \emph{additional} amount of information required to
specify $X=x$ is:
\begin{definition}
\begin{equation}
\begin{split}
\text{KL}(p\Vert{}q)&=-\int{}p(X)\ln{}q(X)\text{d}X-(-\int{}p(X)\ln{}p(X)\text{d}X)\\
=&-\int{}p(X)\ln{}\{\frac{q(X)}{p(X)}\}\text{d}X
\end{split}
\label{eqn:kl}
\end{equation}
\end{definition}
Obviously, \emph{Kullback-Leibler divergence} is not symmetric. Since $(-\ln{}x)''=\frac{1}{x^2}>0$, $-\ln{}x$ is a convex function.
Thus we can apply Jensen inequality to it:
\begin{equation}
\int{}-\ln{}\{\frac{q(X)}{p(X)}\}p(X)\text{d}X\geq{}-\ln{}\int{}p(X)\frac{q(X)}{p(X)}\text{d}X=0
\end{equation}
Obviously, KL$(p\Vert{}q)=0$ only when $q(X)=p(X)$. Besides, we have:
\begin{conclusion}
KL divergence is the \emph{lower bound} for average additional information that
must be transmitted. We can arrive the bound by choosing an efficient
coding scheme.
\end{conclusion}


Both data compression and density estimation are aimed at modelling an
unknown probability distribution. Suppose $q(X)$ is characterized by a
bunch of parameters $\theta$ and we estimate $\theta$ by minimizing KL
divergence. However, $p(X)$ is unknown, we only have its samples
$x_i,i=1,\ldots,N$ and we use this traning set to approximate $p(X)$:
\begin{equation}
\text{KL}(p\Vert{}q)\simeq{}\sum_{i=1}^{N}\{-\ln{}q(x_{i}\vert{}\theta)+\ln{}p(x_i)\}
\label{eqn:klsimeq}
\end{equation}
Note that the second term on the right-hand side of
\eqref{eqn:klsimeq} is indepedent of $\theta$, and the first term is
the negative log likelihood function for $\theta$. Thus we have:
\begin{conclusion}
Minimizing KL divergence $\Leftrightarrow$ Maximizing the likelihood
function.
\end{conclusion}


Given two random variables $X,Y$, they are independent if
$\Pr(X,Y)=\Pr(X)\Pr(Y)$. We can measure how independent they are by:
\begin{definition}
\begin{equation}
\begin{split}
I[X,Y]&\equiv{}\text{KL}(\Pr(X,Y)\Vert{}\Pr(X)\Pr(Y))\\
&=-\int{}\int{}\Pr(X,Y)\ln{}(\frac{\Pr(X)\Pr(Y)}{\Pr(X,Y)})\text{d}X\text{d}Y
\end{split}
\label{eqn:mutualinformation}
\end{equation}
\end{definition}
which is called the \emph{mutual information} between $X$ and $Y$.


Mutual information is related to conditional entropy:
\begin{equation}
\begin{split}
I[X,Y]&=-\int{}\int{}\Pr(X,Y)\ln{}(\frac{\Pr(X)\Pr(Y)}{\Pr(X,Y)})\text{d}X\text{d}Y\\
&=H[X]+H[Y]-(-\int{}\int{}\Pr(X,Y)\ln{}\Pr(X,Y)\text{d}X\text{d}Y)\\
&=H[X]+H[Y]-H[X,Y]\\
&=H[X]-H[X\vert{}Y]=H[Y]-H[Y\vert{}X]\quad{}\text{(by
    \eqref{eqn:condhhh})}
\end{split}
\end{equation}
Thus, the mutual information can also be viewed as the residue in
the uncertainty about $X$ after being told the value of $Y$ (or vice
        versa).

